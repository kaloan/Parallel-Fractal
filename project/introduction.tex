\section{Въведение}
	\begin{Large}
	Целта на проекта е изобразяване на фрактал, използвайки многонишкови изчисления, при решаването на следната задача от "Манделбродов" \ тип:\\
	
	\indent Дефинираме евклидова норма $\left\Vert \cdot \right\Vert$ в комплексната $\mathbb{C}$ така - \ $\left\Vert z \right\Vert=\sqrt{Re(z)^2+Im(z)^2}$. Нека са фиксирани $z_{0}\in\mathbb{C}$, и фунцкия $f:\mathbb{C}\mapsto\mathbb{C}$. За произволна точка от равнината $c\in\mathbb{C}$ търсим дали редицата $\{z_{n}\}_{n=0}^{\infty}$, която ще наричаме орбита на $z_{0}$, получена по следната рекурентна зависимост: 
	\begin{equation} \label{eq:1}
	\forall{n} (z_{n+1}=f(z_{n}))
\end{equation}
притежава следното свойство:
	\begin{equation} \label{eq:2}
	\exists{M_{\in{\mathbb{R}}}} \forall{n} ( \left\Vert z_n \right\Vert \leq M)
\end{equation}
Точките с горното свойство образуват фрактално множество. Изобразяването на точките е в зависимост дали притежават свойството (2), или колко бързо е достигнато достатъчно условие за неналичието му.
\\

\vspace{0.3cm}

Поставени са математическите условията $z_{0}=0$ и $f(z)=e^{z^2+c}$. 
\\
\indent Поставени са програмните условия да е възможно използването на произволен брой нишки в изчисленията, да се изобразява произволен правоъгълник в $\mathbb{C}$, да се създава изображение с произволна резолюция, избор на грануларност. Тези променливи стойности да се задават като командни параметри във форматиран вид, а при липсата на някой от тях се взимат подразбиращи се стойности: максимално една нишка се използва, за изобразяването на $D=\{x,y \ : \ x\in[-2,2] \land y\in[-2,2]\}$, създавайки изображение 640x480.
\newline
\vspace{0.3cm}

Обсъждат се: 
\begin{itemize}
\item Математическите възможности за "лесно" определяне точките, принадлежащи $M$ и тези принадлежащи на $\mathbb{C} \setminus M$.
\item Разпределението на изчисленията и възможните грануларности.
\item Избора на оцветяването.
\end{itemize}

	\end{Large}


