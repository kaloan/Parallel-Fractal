\section{Разпределение на работата}
\begin{Large}
 За решението на задачата трябва да направим еднотипни изчисления на множество от пиксели. Затова ще използваме SPMD. Работата може да се извърши асинхронно. 
 \newline
 При голям брой нишки е възможно да се определят 1-2,които да се грижат само за оцветяването, като тогава ще се наложи синхронизиране - асинхронно подаване на съобщения. Ако се желае моментално записване на информацията в изображението след пресмятането на цвета, то може и да са необходими семафори(в зависимост от това как работи форматът на изображението и дадените функции за обработката му.
 \newline
 Ще разгледаме няколко общи метода на статично разпределение на работата, като ще се опитаме да дадем обща оценка за техните качества - ефикасност, възможност за гранулярност, сложност на имплементация.
 
 \renewcommand\theenumi {\Roman{enumi}}
 \begin{enumerate}{\theenumi}{\Roman{enumi}}
\item Поелементен метод
\par
Да разгледаме как би било добре да се разпредели работата, ако всяка нишка обработва "блокове" от единични пиксели.
В хода на програмата ще се наложи да пазим поне едно от следните: масив за итерации, масив за пиксели, отворено изображение. Такъв тип структури обикновено се реализират чрез разпределение в паметта на едина последователност от данни, образуваща линеен масив от елементите на двумерния, като информацията се пази ред по ред. Имайки предвид това и асоциативността на кеша на процесора, при разглеждане на пикселите "последователно" би било най-добре да правим изчисления за тях ред по ред, като зададем статично циклично работа на нишките по начина показан по-долу.
 
 \hspace{1.5cm}
 \begin{tikzpicture}
\tikzmath{\x = 0.5; \xx=7;}  
 

\fill[green!80!white] (0,0) rectangle (\x,-\x);
\fill[red!80!white] (\x,0) rectangle (2*\x,-\x);
\fill[blue!80!white] (2*\x,0) rectangle (3*\x,-\x);

\fill[green!80!white] (3*\x,0) rectangle (4*\x,-\x);
\fill[red!80!white] (4*\x,0) rectangle (5*\x,-\x);
\fill[blue!80!white] (5*\x,0) rectangle (6*\x,-\x);

\draw[step=0.5cm,gray,very thin] (0,0) grid (6*\x,-\x);


\draw node at (7*\x,-0.5*\x) {...};
\draw node at (3*\x,-1.5*\x) {\begin{small} Стандартен метод\end{small}};




\fill[green!80!white] (0+\xx,0) rectangle (2*\x+\xx,-\x);
\fill[red!80!white] (2*\x+\xx,0) rectangle (4*\x+\xx,-\x);
\fill[blue!80!white] (4*\x+\xx,0) rectangle (6*\x+\xx,-\x);

\draw[step=0.5cm,gray,very thin] (0+7,0) grid (3+7,-0.5);

\draw node at (7*\x+\xx,-0.5*\x) {...};
\draw node at (3*\x+\xx,-1.5*\x) {\begin{small} С гранулярност 2\end{small}};


\draw node at (3*\x,-2.5*\x) {\begin{small} \textcolor{green}{Нишка 1}\end{small}};
\draw node at (12*\x,-2.5*\x) {\begin{small} \textcolor{red}{Нишка 2}\end{small}};
\draw node at (21*\x,-2.5*\x) {\begin{small} \textcolor{blue}{Нишка 3}\end{small}};

\end{tikzpicture}
 
 Възможна е гранулация - например число $k$, така че блоковете да не са от по един, а от по $k$ пиксела. Така големите стойности на $k$ водят до груба гранулация.
 
 \item Регионен метод
 \par
 Желаем да разпределим изображението на правоъгълни региони, като всяка нишка прави изчисления само за зададения ѝ регион. Достигаме много бързо до проблем - какво правим, ако имаме нечетен брой нишки? Единият вариант е да не използваме предоставения максимален брой нишки, но това би било неразумно. Възможно е също да изчислим размера на "нечетния" регион. Отчитайки казаното за кеша би било по-разумно да го разположим с по-дългата част да обхване широчината, а по-късата да е по височината на изображението. Тъй като регионите ще са със сравнително голяма дължина, не би трябвало да имаме проблеми със запазването на информацията в кеша, колкото в предния подход. Заради зависимостта от четност, имплементацията ще се усложни. Този подход не се подлага на гранулярност.
 
 \hspace{1.5cm}
 \begin{tikzpicture}
 \begin{normalsize}


 
 \tikzmath{\x = 0.5;\n=12;\m=16;}
 
 
 \fill[green!80!white] (0,0) rectangle (\m*\x/2,-\n*\x/3);
\fill[red!80!white] (\m*\x/2,0) rectangle (\m*\x,-\n*\x/3);
\fill[blue!80!white] (0,-\n*\x/3) rectangle (\m*\x/2,-2*\n*\x/3);
\fill[purple!80!white] (\m*\x/2,-\n*\x/3) rectangle (\m*\x,-2*\n*\x/3);
\fill[brown!80!white] (0,-2*\n*\x/3) rectangle (\m*\x,-3*\n*\x/3);
\fill[orange!80!white] (\m*\x/2,-2*\n*\x/3) rectangle (\m*\x,-3*\n*\x/3);
 
 \draw[step=\x cm,gray,very thin] (0,0) grid (\m*\x,-\n*\x);
 
 \node at (\m*\x+3,-\x) {\textcolor{green}{Нишка 1}};
 \node at (\m*\x+3,-3*\x) {\textcolor{red}{Нишка 2}};
 \node at (\m*\x+3,-5*\x) {\textcolor{blue}{Нишка 3}};
 \node at (\m*\x+3,-7*\x) {\textcolor{purple}{Нишка 4}};
 \node at (\m*\x+3,-9*\x) {\textcolor{brown}{Нишка 5}};
 \node at (\m*\x+3,-11*\x) {\textcolor{orange}{Нишка 6}};
  \end{normalsize}
\end{tikzpicture}
 
  \item Квадратен метод
  \par
  За него беше споменато на лекции. При $n$ нишки разделяме работата на квадратна мрежа $n \times n$. Разпределянето им по нишки може да стане по следните начини:
\begin{itemize}
\item Произволно - не много добра идея, защото може да се окаже, че някоя от нишките не върши почти нищо, а за сметка на това поне една друга ще трябва да извърши повече изчисления.


\item Циклично - по редове или колони на мрежата. Тук не би трябвало да има значение от разпределението за по-ниските нива на кеша, но с оглед например L3 вероятно е по-добре да разпределим по редове. Получаваме проблем, защото може да се окаже, че нишка може да смята голям свързан регион, където е възможно да има много числа от фракталното множество и да отнеме повече време.


\item Единственост по ред и стълб - подсигуряваме всяка от нишките да изчислява в определена двойка $($\textit{ред,стълб}$)$ по само веднъж. Например задаваме работата последователно на първия ред, след което "превъртаме" последователността и прилагаме за долния ред. Пресмятането на "превъртането" е лесна задача използвайки смятане $(mod \ n)$.

\end{itemize}

   \hspace{1.5cm}
 \begin{tikzpicture}
 \begin{normalsize}


 
 \tikzmath{\x = 0.5;\n=12;\m=15;\th=3;}
 
 
 \fill[green!80!white] (0,0) rectangle (\m*\x/\th,-\n*\x/\th);
\fill[red!80!white] (\m*\x/\th,0) rectangle ++(\m*\x/\th,-\n*\x/\th);
\fill[blue!80!white] (2*\m*\x/\th,0) rectangle ++(\m*\x/\th,-\n*\x/\th);

\fill[blue!80!white] (0,-\n*\x/\th) rectangle ++(\m*\x/\th,-\n*\x/\th);
\fill[green!80!white] (\m*\x/\th,-\n*\x/\th) rectangle ++(\m*\x/\th,-\n*\x/\th);
\fill[red!80!white] (2*\m*\x/\th,-\n*\x/\th) rectangle ++(\m*\x/\th,-\n*\x/\th);

\fill[red!80!white] (0,-2*\n*\x/\th) rectangle ++(\m*\x/\th,-\n*\x/\th);
\fill[blue!80!white] (\m*\x/\th,-2*\n*\x/\th) rectangle ++(\m*\x/\th,-\n*\x/\th);
\fill[green!80!white] (2*\m*\x/\th,-2*\n*\x/\th) rectangle ++(\m*\x/\th,-\n*\x/\th);
 
 \draw[step=\x cm,gray,very thin] (0,0) grid (\m*\x,-\n*\x);
 
 \node at (\m*\x+3,-\x) {\textcolor{green}{Нишка 1}};
 \node at (\m*\x+3,-4*\x) {\textcolor{red}{Нишка 2}};
 \node at (\m*\x+3,-7*\x) {\textcolor{blue}{Нишка 3}};
 
 \node at (\m*\x+3,-\n*\x+\x) {Единственост по ред и стълб};
\end{normalsize}
\end{tikzpicture}
  
  Този метод е по-разумент от другите два. Също така лесно може да се въведе гранулярност като число $k$ и мрежата да се разпада на $kn \times kn$(т.е. големите числа водят до фина грануларност). Тогава обаче последният начин е неприложим. Този метод не е сложен за имплементация.
  \newline
  %(в зависимост от четността на дължината/широчината на изображението може и да се получат правоъгълници по някой от 
   \hspace{1.5cm}
 \begin{tikzpicture}
 \begin{normalsize}


 
 \tikzmath{\x = 0.5;\n=12;\m=18;\th=3;\g=3;}
 
 
 \fill[green!80!white] (0,0) rectangle ++(\m*\x/\th/\g,-\n*\x);
\fill[green!80!white] (\m*\x/\th,0) rectangle ++(\m*\x/\th/\g,-\n*\x);
\fill[green!80!white] (2*\m*\x/\th,0) rectangle ++(\m*\x/\th/\g,-\n*\x);

\fill[red!80!white] (2*\g*\x/\th,0) rectangle ++(\m*\x/\th/\g,-\n*\x);
\fill[red!80!white] (\m*\x/\th+2*\g*\x/\th,0) rectangle ++(\m*\x/\th/\g,-\n*\x);
\fill[red!80!white] (2*\m*\x/\th+2*\g*\x/\th,0) rectangle ++(\m*\x/\th/\g,-\n*\x);

\fill[blue!80!white] (4*\g*\x/\th,0) rectangle ++(\m*\x/\th/\g,-\n*\x);
\fill[blue!80!white] (\m*\x/\th+4*\g*\x/\th,0) rectangle ++(\m*\x/\th/\g,-\n*\x);
\fill[blue!80!white] (2*\m*\x/\th+4*\g*\x/\th,0) rectangle ++(\m*\x/\th/\g,-\n*\x);

 
 \draw[step=\x cm,gray,very thin] (0,0) grid (\m*\x,-\n*\x);
 \draw[step=2*\x cm,pink,very thin] (0.2,0.2) grid (\m*\x-0.2,-\n*\x-0.2);
 
 \node at (\m*\x+3,-\x) {\textcolor{green}{Нишка 1}};
 \node at (\m*\x+3,-4*\x) {\textcolor{red}{Нишка 2}};
 \node at (\m*\x+3,-7*\x) {\textcolor{blue}{Нишка 3}};
 
 \node[align=center] at (\m*\x+3,-\n*\x+2*\x) {Цикличноно разпределение \\ с гранулярност 2};
\end{normalsize}
\end{tikzpicture}

  \item Линеен метод
  \par
  Това е реализираният метод в проекта. С оглед казаното по-горе, разделяме изчисленията ред по ред. Всеки ред се изчислява от единствена нишка, като разпределянето е циклично, започвайки от първата създадена. Малко вероятно е една и съща нишка да попадне многократно на по-големи(в сравнение с другите нишки) участъци от по-сложни изчисления,т.е. би трябвало работата да е добре разпределена.
  \newline
	   \hspace{1.5cm}
 \begin{tikzpicture}
 \begin{normalsize}


 
 \tikzmath{\x = 0.5;\n=6;\m=12;\th=3;\d=4;}
 
 
 \fill[green!80!white] (0,0) rectangle ++(\m*\x,-\x);
\fill[red!80!white] (0,-\x) rectangle ++(\m*\x,-\x);
\fill[blue!80!white] (0,-2*\x) rectangle ++(\m*\x,-\x);
  \fill[green!80!white] (0,-3*\x) rectangle ++(\m*\x,-\x);
\fill[red!80!white] (0,-4*\x) rectangle ++(\m*\x,-\x);
\fill[blue!80!white] (0,-5*\x) rectangle ++(\m*\x,-\x);

 \draw[step=\x cm,gray,very thin] (0,0) grid (\m*\x,-\n*\x);
 \node[align=center] at (\m*\x/2,-\n*\x-\x) {Прост линеен метод}; 
 
 
 
  \fill[green!80!white] (\d+\m*\x,0) rectangle ++(\m*\x,-2*\x);
\fill[red!80!white] (\d+\m*\x,-2*\x) rectangle ++(\m*\x,-2*\x);
\fill[blue!80!white] (\d+\m*\x,-4*\x) rectangle ++(\m*\x,-2*\x);
 
 \draw[step=\x cm,gray,very thin] (\d+\m*\x,0) grid (\d+2*\m*\x,-\n*\x);
 
 \node at (\m*\x+\d/2,-\x) {\textcolor{green}{Нишка 1}};
 \node at (\m*\x+\d/2,-3*\x) {\textcolor{red}{Нишка 2}};
 \node at (\m*\x+\d/2,-5*\x) {\textcolor{blue}{Нишка 3}};
 
 \node[align=center] at (3*\m*\x/2+\d,-\n*\x-\x) {Линеен метод с гранулярност 2};
\end{normalsize}
\end{tikzpicture}  
\newline
  Методът позволява грануларност - някакво чиско $k$, т.ч. изчисленията стават не ред по ред, а в групи от $k$ реда(т.е. базовият случай е k=1). Така по-големи $k$ водят до по-груба гранулярност. Методът е елементарен за имплементация.
 \end{enumerate}
\renewcommand\theenumi {\arabic{enumi}}

\end{Large}